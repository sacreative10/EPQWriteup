
\documentclass[12pt, a4paper, english, twocolumn]{article}

\usepackage[l2tabu,orthodox]{nag}  % force newer (and safer) LaTeX commands
\usepackage[utf8]{inputenc}        % set character set to support some UTF-8
                                   %   (unicode). Do NOT use this with
                                   %   XeTeX/LuaTeX!
\usepackage{babel}                 % multi-language support
\usepackage{sectsty}               % allow redefinition of section command formatting
\usepackage{tabularx}              % more table options
\usepackage{titling}               % allow redefinition of title formatting
\usepackage{imakeidx}              % create and index of words
\usepackage{xcolor}                % more colour options
% \usepackage{enumitem}              % more list formatting options
\usepackage{tocloft}               % redefine table of contents, new list like objects
\usepackage{amsmath}                                    % extensive math options
\usepackage{amssymb}                                    % special math symbols
% \usepackage[Gray,squaren,thinqspace,thinspace]{SIunitsx} % elegant units
\usepackage{listings}                                   % source code
\usepackage{tikz}
\usepackage[centering,noheadfoot,margin=1in]{geometry}
\setlength{\parindent}{0cm}
\setlength{\parskip}{2ex plus 0.5ex minus 0.2ex} % whitespace between paragraphs

% set TOC margins
\setlength{\cftbeforesecskip}{15pt} % skip in TOC
% also possible (incl. variants): \setlength{\cftbeforechapskip}{10pt}
\usepackage{parskip}
\usepackage{graphicx}
% redefine \maketitle command with nicer formatting
\pretitle{
  \begin{flushright}         % align text to right
    \fontsize{40}{60}        % set font size and whitespace
    \usefont{OT1}{phv}{b}{n} % change the font to bold (b), normally shaped (n)
                             %   Helvetica (phv)
    \selectfont              % force LaTeX to search for metric in its mapping
                             %   corresponding to the above font size definition
}
\posttitle{
  \par                       % end paragraph
  \end{flushright}           % end right align
  \vskip 0.5em               % add vertical spacing of 0.5em
}
\preauthor{
  \begin{flushright}
    \large                   % font size
    \lineskip 0.5em          % inter line spacing
    \usefont{OT1}{phv}{m}{n}
}
\postauthor{
  \par
  \end{flushright}
}
\predate{
  \begin{flushright}
  \large
  \lineskip 0.5em
  \usefont{OT1}{phv}{m}{n}
}
\postdate{
  \par
  \end{flushright}
}
% NEEDS to be before hyperref, cleveref and autonum
% number figures, tables and equations within the sections
\numberwithin{equation}{section}
\numberwithin{figure}{section}
\numberwithin{table}{section}
% references and annotation, citations
\usepackage[small,bf,hang]{caption}        % captions
\usepackage{subcaption}                    % adds subfigure & subcaption
\usepackage{sidecap}                       % adds side captions
\usepackage{hyperref}                      % add hyperlinks to references
\hypersetup{
  colorlinks=true,
  linkcolor=black,
  filecolor=magenta,
  urlcolor=cyan,
}
\usepackage[noabbrev,nameinlink]{cleveref} % better references than default \ref
% \usepackage{autonum}                       % only number referenced equations
\usepackage{url}                           % urls
% \usepackage{cite}                          % well formed numeric citations

\usepackage{lipsum}

\usepackage{subfiles} % Best loaded last in the preamble

\usepackage[style=nature]{biblatex}
\addbibresource{Library.bib}
\usepackage{microtype}


\begin{document}
\begin{titlepage}
   \begin{center}
       \vspace*{1cm}

       \textbf{Physically Based Rendering}

       \vspace{0.5cm}
        The art of the science of Light

        $ L_{o}(p, \omega_{0}) = L_{e}(p, \omega_{0}) + \displaystyle\int_{S^2}^{} f(p, \omega_{0}, \omega_{i}) L_{i}(p, \omega_{i}) |cos \theta_{i}|d \omega_{i} $
            
       \vspace{1.5cm}

       \textbf{Sohaib Alam}

       \vfill
            
       A thesis presented for an Extended Project Qualification
            
       \vspace{0.8cm}
     
            
       St Bartholomew's School\\
       United Kingdom\\
       \today
            
   \end{center}

\end{titlepage}


\onecolumn
\tableofcontents

\newpage

\begin{abstract}
This thesis covers a method of Photorealistic Rendering called Physically Based Ray-Tracing (or just Ray-Tracing). Within the introduction, 
exists a brief history of the field, as well as the current state of the field. The research review covers the literature that was used to
create this project. The design overview covers the design of the ray-tracer. The development section covers the implementation of the ray-tracer.
Lastly, the conclusion and evaluation section covers the evaluation of the project, including the presentation.
\end{abstract}

\twocolumn



\subfile{sections/Introduction}
\subfile{sections/Research_Review}
\subfile{sections/Design_Overview}
\subfile{sections/Development}
\subfile{sections/Conclusion_Evaluation}
\cleardoublepage
\printbibliography[heading=bibintoc]
\onecolumn
\subfile{sections/Appendix}


\end{document}
