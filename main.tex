
\documentclass[12pt, a4paper, english, twocolumn]{article}

\usepackage[l2tabu,orthodox]{nag}  % force newer (and safer) LaTeX commands
\usepackage[utf8]{inputenc}        % set character set to support some UTF-8
                                   %   (unicode). Do NOT use this with
                                   %   XeTeX/LuaTeX!
\usepackage{babel}                 % multi-language support
\usepackage{sectsty}               % allow redefinition of section command formatting
\usepackage{tabularx}              % more table options
\usepackage{titling}               % allow redefinition of title formatting
\usepackage{imakeidx}              % create and index of words
\usepackage{xcolor}                % more colour options
% \usepackage{enumitem}              % more list formatting options
\usepackage{tocloft}               % redefine table of contents, new list like objects
\usepackage{amsmath}                                    % extensive math options
\usepackage{amssymb}                                    % special math symbols
% \usepackage[Gray,squaren,thinqspace,thinspace]{SIunitsx} % elegant units
\usepackage{listings}                                   % source code
\usepackage{tikz}
\usepackage[centering,noheadfoot,margin=1in]{geometry}
\setlength{\parindent}{0cm}
\setlength{\parskip}{2ex plus 0.5ex minus 0.2ex} % whitespace between paragraphs

% set TOC margins
\setlength{\cftbeforesecskip}{15pt} % skip in TOC
% also possible (incl. variants): \setlength{\cftbeforechapskip}{10pt}
\usepackage{parskip}
\usepackage{graphicx}
% redefine \maketitle command with nicer formatting
\pretitle{
  \begin{flushright}         % align text to right
    \fontsize{40}{60}        % set font size and whitespace
    \usefont{OT1}{phv}{b}{n} % change the font to bold (b), normally shaped (n)
                             %   Helvetica (phv)
    \selectfont              % force LaTeX to search for metric in its mapping
                             %   corresponding to the above font size definition
}
\posttitle{
  \par                       % end paragraph
  \end{flushright}           % end right align
  \vskip 0.5em               % add vertical spacing of 0.5em
}
\preauthor{
  \begin{flushright}
    \large                   % font size
    \lineskip 0.5em          % inter line spacing
    \usefont{OT1}{phv}{m}{n}
}
\postauthor{
  \par
  \end{flushright}
}
\predate{
  \begin{flushright}
  \large
  \lineskip 0.5em
  \usefont{OT1}{phv}{m}{n}
}
\postdate{
  \par
  \end{flushright}
}
% NEEDS to be before hyperref, cleveref and autonum
% number figures, tables and equations within the sections
\numberwithin{equation}{section}
\numberwithin{figure}{section}
\numberwithin{table}{section}
% references and annotation, citations
\usepackage[small,bf,hang]{caption}        % captions
\usepackage{subcaption}                    % adds subfigure & subcaption
\usepackage{sidecap}                       % adds side captions
\usepackage{hyperref}                      % add hyperlinks to references
\hypersetup{
  colorlinks=true,
  linkcolor=black,
  filecolor=magenta,
  urlcolor=cyan,
}
\usepackage[noabbrev,nameinlink]{cleveref} % better references than default \ref
% \usepackage{autonum}                       % only number referenced equations
\usepackage{url}                           % urls
% \usepackage{cite}                          % well formed numeric citations

\usepackage{lipsum}

\usepackage{subfiles} % Best loaded last in the preamble

\usepackage[style=nature]{biblatex}
\addbibresource{Library.bib}
\usepackage{microtype}


\begin{document}
\begin{titlepage}
   \begin{center}
       \vspace*{1cm}

       \textbf{Physically Based Rendering}

       \vspace{0.5cm}
        The art of the science of Light

        $ L_{o}(p, \omega_{0}) = L_{e}(p, \omega_{0}) + \displaystyle\int_{S^2}^{} f(p, \omega_{0}, \omega_{i}) L_{i}(p, \omega_{i}) |cos \theta_{i}|d \omega_{i} $
            
       \vspace{1.5cm}

       \textbf{Sohaib Alam}

       \vfill
            
       A thesis presented for an Extended Project Qualification
            
       \vspace{0.8cm}
     
            
       St Bartholomew's School\\
       United Kingdom\\
       \today
            
   \end{center}

\end{titlepage}


\onecolumn
\tableofcontents

\newpage

\begin{abstract}
  Computer graphics is a nascent field that has seen 
  rapid growth in the past few decades, and with 
  the advent of CPUs and GPUs, being able to compute 
  complex algorithms in real-time, has allowed for
  the development of rendering algorithms that can
  accurately model the behavior of light in the real
  world. This project aims to explore one such algorithm,
  Physically Based Ray-tracing, and its usage in the creation
  of realistic images. Such algorithms are used in the
  creation of computer-generated imagery in movies and 
  video games, and produce images that are almost 
  indistinguishable from real-world photographs.
  This thesis will explore the theory behind Physically
  Based Ray-tracing, its implementation, and the results.
  The project found that Physically Based Ray-tracing
  is a powerful tool that can be used to create realistic
  images, but is computationally expensive and requires
  a non-trivial computations to produce images in a
  reasonable amount of time.
\end{abstract}

\twocolumn



\subfile{sections/Introduction}
\subfile{sections/Research_Review}
\subfile{sections/Design_Overview}
\subfile{sections/Development}
\subfile{sections/Conclusion_Evaluation}
\cleardoublepage
\printbibliography[heading=bibintoc]
\onecolumn
\subfile{sections/Appendix}


\end{document}
