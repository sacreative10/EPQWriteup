
\documentclass[../main.tex]{subfiles}

\begin{document}
\subsection{History and Current State of the Field}
Unsurprisingly, Physically Based Rendering (PBR), is a relatively nascent field (only being studied since the 70s), and as the field has advanced to cleverer solution to increasingly difficult problems. 
For example, in the 70s, the biggest problem to solve was the lack of memory available to computers (1 MB at its rare), where physical accuracy was not biggest focus, but to speed up how long an image took to render \cite{pharr_physically_2016}. This was achieved by taking a subset of the entire scene representation into memory, however this caused problems with global illumination algorithms, as this (the name might suggest) required a larger (or the entire) scene to be loaded into memory. So, with speed, creators of such graphics decided that geometric accuracy was more important then realism of light. Because of this, many people though at the time that realism was not as important as artistic merit, where the computer graphics was used as an aid, rather than a tooling which is essential from the ground up. 

The best example of this is the rise of Computer Generated Imagery (CGI) within the film industry, where some of the biggest blockbuster releases, leveraged the power of computers to render backgrounds and add complex elements to a scene, or augment a character's physical appearance (e.g. Terminator 2) \cite{bramesco_terminator_2021}.

When computers did become advanced enough to render scenes in full, many film studios adopted the physically based rendering pipeline into their film-making process. A good case study is Pixar, who have been propellers of this field into the mainstream. Their \textit{Renderman} renderer has used the photorealistic algorithm, and have won several awards for their films. They also contribute to the SIGGRAPH community, which is another essential source in this field.

However, as Jim Blinn states: "as technology advances, rendering time remains constant". This observes that as technology advances, people's demand for "better" and "more realistic" images increases, rather than being content with current standards, and rendering those scenes faster \cite{pharr_physically_2016}.

\subsubsection{Relevant Literature}
Physically Based Rendering: From Theory to Implementation 3rd edition \cite{pharr_physically_2016} is perhaps the best book on this topic. It was written by ex-NVdia and ex-Google employees, who specialise in image rendering. It is the only consolidatory source in this field and provides excellent resources for further research (which proved useful for certain topics within my EPQ), as well as provide exercises to get a better understanding of the topic. If I have to mention any drawbacks about this literature, would be its age. Written in 2016, this book lacks to mention any alternatives, or cover use of specialist equipment, like GPUs. However, do note this problem is remedied within the 4th edition of the book.

   Glassner's introduction to the ray-tracing algorithm provides an excellent insight into the basics of the algorithm. Though from the 80s, it still is a excellent foundation for more complex and modern algorithms \cite{glassner_introduction_1989}  
   
   Highner's book on numerical algorithms, isn't related to ray-tracing or PBR, but provides an excellent bank of algorithms and "tricks" to solving complex problems, such as the Monte-Carlo Integration algorithm \cite{higham_accuracy_2002}.

   Glassner's book on "Graphics Gems" is also a very useful resource and similar Highner's book, provides insightful knowledge into algorithms referenced in Physically based Rendering 3rd ed \cite{ap_professional_firm_ap_1995}.

   Woop, Benthin and Wald's paper on ray-triangle intersections is a essential resource, as provides a novel algorithm in solving the problem of ray-triangle intersections. Without it, it would be near impossible to render complex objects which aren't quadrics \cite{woop_watertight_2013}.

   Shirley's 3-part series on ray-tracing are an excellent method, from which I dipped my feet into the world of PBR. While the premise of the series might be considered by most as simple, it still is an important resource on solving problems simply rather than robustly \cite{peter_shirley_trevor_david_black_steve_hollasch_ray_nodate}.

   Cohen and Kappauf's paper on colour theory was also useful when making the camera of the ray-tracer, it broke down the complicated Commission Internationale de l’Éclairage (CIE) standards on colour \cite{cohen_color_1985}.
\end{document}
