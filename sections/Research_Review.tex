
\documentclass[../main.tex]{subfiles}

\begin{document}
\subsection{History and Current State of the Field}
Unsurprisingly, Physically Based Rendering (PBR), is a relativly nascent field (only being studied since the 70s), and as the field has advanced to cleverer solution to increasingly difficult problems. 
For example, in the 70s, the biggest problem to solve was the lack of memory available to computers (1 MB at its rare), where physical accuracy was not biggest focus, but to speed up how long an image took to render \cite{pharrPhysicallyBasedRendering2016}. As computers became less expensive and more powerful, more demanding scenes could be rendered.

The best example of this is the rise of Computer Generated Imagery (CGI) within the film industry, where some of the biggest blockbuster releases, leveraged the power of computers to render backgrounds and add complex elements to a scene, or augment a character's physical appearance (e.g. Terminator 2).

However, as Jim Blinn states: "as technology advances, rendering time remains constant". This observes that as technology advances, people's demand for "better" and "more realistic" images increases, rather than being content with current standards, and rendering those scenes faster \cite{pharrPhysicallyBasedRendering2016}.
\subsubsection{Research}
\end{document}
