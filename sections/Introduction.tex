
\documentclass[../main.tex]{subfiles}

\begin{document}
\section{Introduction}
Ever since the advent of computers, there has been a concerted effort to imitate the behaviours we notice in our world, as an endeavour to solve problems, that couldn't be solved without computers. Perhaps, in failing to do so, one can discover other avenues to solve a related (or unrelated) problem. 
For example, Craig Reynolds in his seminal paper \cite{reynoldsFlocksHerdsSchools1987} is an excellent case study, of such an example. Reynolds defined three very simple rules pertaining to the behaviours of birds in a flock. These rules are: 
\begin{itemize}
  \item Alignment: the birds will steer towards the average heading of their peers.
  \item Cohesion: the birds will steer towards the center of the mass of its peers.  
  \item Seperation: the birds will steer away from colliding with their peers.
\end{itemize}
These simple laws, though primitive, produce a "realistic" approximation of how birds behave in actuality, so much so, that the US Army uses this algorithm, for their UAV-UGV programs \cite{saskaCoordinationNavigationHeterogeneous2014}.

In essence, computer simulation provides a "good enough" approximation for the theories encompassing the real world, crafted by mathematicians and physicists alike.
\subsection{Motivation}
Rendering is the process of producing an image from a description of a 3D scene. This daunting task can be easily understood by asking: Given a room configuration 
\end{document} 
